\documentclass[10pt,twocolumn]{article}
\usepackage[margin=0.5in]{geometry}
\usepackage{mathtools}
\usepackage{amssymb} 
\title{\textbf{Probability Assignment}}
\author{Sinkona Chinthamalla}

\providecommand{\pr}[1]{\ensuremath{\Pr\left(#1\right)}}
\providecommand{\brak}[1]{\ensuremath{\left(#1\right)}}
\providecommand{\cbrak}[1]{\ensuremath{\left\{#1\right\}}}
\providecommand{\lcbrak}[1]{\ensuremath{\left\{#1\right.}}
\providecommand{\rcbrak}[1]{\ensuremath{\left.#1\right\}}}

\begin{document}
\maketitle

\textbf {12.13.1.10} \\
A black and a red dice are rolled.
\begin{enumerate}
\item Find the conditional probability of obtaining a sum greater than 9, given that the black die resulted in a 5.
\item Find the conditional probability of obtaining the sum 8, given that the red die resulted in a number less than 4.
\end{enumerate}

\subsection*{Solution}
\begin{enumerate}
\item  {\em The Uniform Distribution: }Let $X_i \in \cbrak{1,2,3,4,5,6}, i = 1,2,$ be the random variables representing the outcome for each die.  Assuming the dice to be fair, the probability mass function (pmf) is expressed as 
\begin{align}
p_{X_i}(n) = \pr{X_i = n} = 
\begin{cases}
\frac{1}{6} & 1 \le n \le 6
\\
0 & otherwise
\end{cases}
\end{align}
The desired outcome is
\begin{align}
X &= X_1 + X_2, \\
\implies X &\in \cbrak{1,2,\dots,12}
\end{align}

\item {\em Convolution: } \\
From (2)
\begin{align}
p_X(n) &= \pr{X_1 + X_2 > n} = \pr{X_1  > n -X_2} \\
&= \sum_{k}^{}\pr{X_1  > n -k | X_2 = k}p_{X_2}(k)
\end{align}
after unconditioning. $\because X_1$ and $X_2$ are independent.
Then,
\begin{align}
p_X(n) &= \pr{X_1 + X_2 > 9} = \pr{X_1  > 9 -X_2} \\
&= \pr{X_1  > 9 -k | X_2 = k}p_{X_2}(k) \\
&= \frac{1}{6} \pr{X_1  > 9 -5 | X_2 = 5} \\
&= \frac{1}{6} \pr{X_1 > 4} \\
&= \frac{1}{6} (\pr{X_1 = 5} + \pr{X_1 = 6}) \\
&= \frac{2}{36}
\end{align}
Conditional probability is given by,
\begin{align}
\begin{split}
\Pr{\brak{\brak{X_1 + X_2 > 9}|\brak{X_2=5}}} &= 
\frac{\frac{2}{36}}{\frac{1}{6}}\\
&=\frac{1}{3}
\end{split}
\end{align}  
Hence the probability of obtaining a sum greater than 9, when black die resulted in a 5 is $\frac{1}{3}$. 

\newpage
\item  {\em The Uniform Distribution: }Let $X_i \in \cbrak{1,2,3,4,5,6}, i = 1,2,$ be the random variables representing the outcome for each die. 
\item {\em Convolution: } \\
From (2)
\begin{align}
p_X(n) &= \pr{X_1 + X_2 = n} = \pr{X_1  = n -X_2} \\
&= \sum_{k}^{}\pr{X_1  = n -k | X_2 = k}p_{X_2}(k)
\end{align}
after unconditioning. $\because X_1$ and $X_2$ are independent.
Then,
\begin{align}
p_X(n) &= \pr{X_1 + X_2 = 8} = \pr{X_1  = 8 -X_2} \\
&= \pr{X_1  = 8 -k | X_2 < k}p_{X_2}(k) \\
&= \frac{1}{6} \pr{X_1  = 8 -k | X_2 < 4} \\
&= \frac{1}{6} (\pr{X_1 = 5} + \pr{X_1 = 6}) \\
&= \frac{2}{36}
\end{align}
Conditional probability is given by,
\begin{align}
\begin{split}
\Pr{\brak{\brak{X_1 + X_2 = 8}|\brak{X_2<4}}} &= 
\frac{\frac{2}{36}}{\frac{3}{6}}\\
&=\frac{1}{9}
\end{split}
\end{align}  

Hence the probability of obtaining the sum 8 when a number is less than 4 is $\frac{1}{9}$. 

\end{enumerate}
\end{document}
