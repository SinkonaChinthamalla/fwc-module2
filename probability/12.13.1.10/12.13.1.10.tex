\documentclass[10pt,twocolumn]{article}
\usepackage[margin=0.5in]{geometry}
\usepackage{mathtools}
\usepackage{amssymb} 
\title{\textbf{Probability Assignment}}
\author{Sinkona Chinthamalla}

\providecommand{\pr}[1]{\ensuremath{\Pr\left(#1\right)}}
\providecommand{\brak}[1]{\ensuremath{\left(#1\right)}}
\providecommand{\cbrak}[1]{\ensuremath{\left\{#1\right\}}}
\providecommand{\lcbrak}[1]{\ensuremath{\left\{#1\right.}}
\providecommand{\rcbrak}[1]{\ensuremath{\left.#1\right\}}}

\begin{document}
\maketitle

\textbf {12.13.1.10} \\
A black and a red dice are rolled.
\begin{enumerate}
\item Find the conditional probability of obtaining a sum greater than 9, given that the black die resulted in a 5.
\item Find the conditional probability of obtaining the sum 8, given that the red die resulted in a number less than 4.
\end{enumerate}

\subsection*{Solution}
Let $X_i \in \cbrak{1,2,3,4,5,6}, i = 1,2,$ be the random variables representing the outcome for each die.
\begin{enumerate}
\item Since $X_1$ and $X_2$ are independent,
\begin{align}
p_X(n) &= \pr{X_1 + X_2 > 9} = \pr{X_1  > 9 -X_2} \\
&= \pr{X_1  > 9 -k | X_2 = k}p_{X_2}(k) \\
&= \frac{1}{6} \pr{X_1  > 9 -5 | X_2 = 5} \\
&= \frac{1}{6} \pr{X_1 > 4} \\
&= \frac{1}{6} (\pr{X_1 = 5} + \pr{X_1 = 6}) \\
&= \frac{2}{36}
\end{align}
Conditional probability of event $(X_1 > 4)$ given that $(X_2 = 5)$ has occurred is, \\
$\Pr{\brak{\brak{X_1 > 4}|\brak{X_2=5}}}$ 
\begin{align}
&=\frac{\Pr{((X_1 > 4),(X_2=5))}}{\Pr{\brak{X_2=5}}}\\
&=\frac{\frac{2}{36}}{\frac{1}{6}}\\
&=\frac{1}{3}
\end{align}  
Hence the probability of obtaining a sum greater than 9, when black die resulted in a 5 is $\frac{1}{3}$. 

\item Since $X_1$ and $X_2$ are independent,
\begin{align}
p_X(n) &= \pr{X_1 + X_2 = 8} = \pr{X_1  = 8 -X_2} \\
&= \pr{X_1  = 8 -k | X_2 < k}p_{X_2}(k) \\
&= \frac{1}{6} \pr{X_1  = 8 -k | X_2 < 4} \\
&= \frac{1}{6} (\pr{X_1 = 5} + \pr{X_1 = 6}) \\
&= \frac{2}{36}
\end{align}
Conditional probability of event $((X_1 = 5)+ (X_1 = 6))$ given that $(X_2 < 4)$ has occurred is, \\
$\Pr{\brak{\brak{(X_1 = 5) + (X_1 = 6)}|\brak{X_2<4}}}$ 
\begin{align}
&= \frac{\Pr{(((X_1 = 5) + (X_1 = 6)),(X_2 < 4))}}{\Pr{\brak{X_2 < 4}}} \\
&= \frac{\frac{2}{36}}{\frac{3}{6}}\\
&= \frac{1}{9}
\end{align} 
Hence the probability of obtaining the sum 8 when a number is less than 4 is $\frac{1}{9}$. 
\end{enumerate}
\end{document}
