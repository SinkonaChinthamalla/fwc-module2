\documentclass[10pt,twocolumn]{article}
\usepackage[margin=0.5in]{geometry}
\usepackage{mathtools}
\usepackage{amssymb} 
\title{\textbf{Probability Assignment}}
\author{Sinkona Chinthamalla}

\providecommand{\pr}[1]{\ensuremath{\Pr\left(#1\right)}}
\providecommand{\brak}[1]{\ensuremath{\left(#1\right)}}
\providecommand{\cbrak}[1]{\ensuremath{\left\{#1\right\}}}
\providecommand{\lcbrak}[1]{\ensuremath{\left\{#1\right.}}
\providecommand{\rcbrak}[1]{\ensuremath{\left.#1\right\}}}

\begin{document}
\maketitle

\textbf {12.13.1.10} \\
A black and a red dice are rolled.
\begin{enumerate}
\item Find the conditional probability of obtaining a sum greater than 9, given that the black die resulted in a 5.
\item Find the conditional probability of obtaining the sum 8, given that the red die resulted in a number less than 4.
\end{enumerate}

\subsection*{Solution}
Let $X_1$ be a random variable representing the outcome of black die and $X_2$ be a random variable representing the outcome of red die.
\begin{enumerate}
\item 
Given, \\
Black die resulted in a 5. 

The probability of obtaining a sum greater than 9 is given by,
\begin{align}
\Pr{\brak{\brak{X_2 > 4},\brak{X_1=5}}}
&= \pr{X_1 + X_2 > 9} \\
&= \pr{X_2  > 9 -X_1} \\
&= \pr{X_2  > 9 -k | X_1 = k}p_{X_1}(k) \\
&= \frac{1}{6} \pr{X_2  > 9 -5 | X_1 = 5} \\
&= \frac{1}{6} \pr{X_2 > 4} \\
&= \frac{1}{6} (\pr{X_2 = 5} + \pr{X_2 = 6}) \\
&= \frac{2}{36}
\end{align}
Conditional probability of event $(X_2 > 4)$ given that $(X_1 = 5)$ has occurred is, 
\begin{align}
\Pr{\brak{\brak{X_2 > 4}|\brak{X_1=5}}}
&=\frac{\Pr{((X_2 > 4),(X_1=5))}}{\Pr{\brak{X_1=5}}}\\
&=\frac{\frac{2}{36}}{\frac{1}{6}}\\
&=\frac{1}{3}
\end{align}  
Hence the conditional probability of obtaining a sum greater than 9, when black die resulted in a 5 is $\frac{1}{3}$. 

\newpage
\item Given, \\
Red die resulted in a number less than 4. 

The probability of obtaining the sum 8 is given by,
\begin{align}
\Pr{\brak{\brak{X_1 + X_2 = 8},\brak{X_2<4}}} 
&= \pr{X_1 + X_2 = 8} \\
&= \pr{X_1  = 8 -X_2} \\
&= \pr{X_1  = 8 -k | X_2 < k}p_{X_2}(k) \\
&= \frac{1}{6} \pr{X_1  = 8 -k | X_2 < 4} \\
&= \frac{1}{6} (\pr{X_1 = 5} + \pr{X_1 = 6}) \\
&= \frac{2}{36}
\end{align}
Conditional probability of event $(X_1 + X_2 = 8)$ given that $(X_2 < 4)$ has occurred is, 
\begin{align}
\Pr{\brak{\brak{X_1 + X_2 = 8}|\brak{X_2<4}}} 
&= \frac{\Pr{((X_1 + X_2 = 8),(X_2 < 4))}}{\Pr{\brak{X_2 < 4}}} \\
&= \frac{\frac{2}{36}}{\frac{3}{6}}\\
&= \frac{1}{9}
\end{align} 
Hence the probability of obtaining the sum 8 when a number is less than 4 is $\frac{1}{9}$. 
\end{enumerate}
\end{document}
