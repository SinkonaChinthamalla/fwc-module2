\documentclass[10pt,twocolumn]{article}
\usepackage[margin=0.5in]{geometry}
\usepackage{mathtools}
\title{\textbf{Probability Assignment}}
\author{Sinkona Chinthamalla}

\providecommand{\pr}[1]{\ensuremath{\Pr\left(#1\right)}}
\providecommand{\brak}[1]{\ensuremath{\left(#1\right)}}

\begin{document}
\maketitle

\textbf {12.13.1.10} \\
A black and a red dice are rolled.
\begin{enumerate}
\item Find the conditional probability of obtaining a sum greater than 9, given that the black die resulted in a 5.
\item Find the conditional probability of obtaining the sum 8, given that the red die resulted in a number less than 4.
\end{enumerate}

\subsection*{Solution}
\begin{enumerate}
\item 
Let $X_1 \in \brak{1,2,3,4,5,6}$ be a random variable representing the outcomes of red die.\\
Let $X_2 \in \brak{1,2,3,4,5,6}$ be a random variable representing the outcomes of black die. \\
We need the conditional probability of event $(X_1 + X_2 > 9)$ given that $(X_2 = 5)$ has occurred.
$\Pr{\brak{\brak{X_1 + X_2 > 9}|\brak{X_2=5}}}$
\begin{align}
=\frac{\Pr{((X_1 + X_2 > 9)(X_2=5))}}{\Pr{\brak{X_2=5}}}
\end{align}
We have that,
\begin{align}
p_{X_i}(n) = \pr{X_i = n} = 
\begin{cases}
\frac{1}{6} & 1 \le n \le 6 \\
0 & otherwise
\end{cases}
\end{align}
Therefore using equation (2) we can write that, 
\begin{align}
\Pr{(X_2=5)} = \frac{1}{6}
\end{align}
From binomial distribution we can write ,
\begin{align}
\Pr{(X_1 + X_2 > 9)}&= \Pr{(X_1 + X_2 =10)}+\Pr{(X_1 + X_2=11)} \\ 
&= \binom{2}{1} \brak {\frac{1}{6}}\brak {\frac{5}{6}}+\binom{2}{2} \brak {\frac{1}{6}}^2\\
&= \frac{11}{36}
\end{align}
The event ${((X_1 + X_2 > 9)(X_2=5))}$ is such that the sum is greater than 9 with a number 5. \\
There are only two possible cases \{5,5\} and \{6,5\} out of 36 possible cases.\\
Hence,
\begin{align}
\Pr{((X_1 + X_2 > 9)(X_2=5))}=\frac{2}{36}
\end{align}
Substituting equations (3), (7) in (1), we get
\begin{align}
\begin{split}
\Pr{\brak{\brak{X_1 + X_2 > 9}|\brak{X_2=5}}} &= 
\frac{\frac{2}{36}}{\frac{1}{6}}\\
&=\frac{1}{3}
\end{split}
\end{align}  
Hence the probability of obtaining a sum greater than 9, with black die in a 5 is $\frac{1}{3}$. 

\item 
Let $X_1 \in \brak{1,2,3,4,5,6}$ be a random variable representing the outcomes of black die.\\
Let $X_2 \in \{1,2,3,4,5,6\}$ be a random variable representing the outcomes of red die. \\
We need the conditional probability of event $(X_1 + X_2 = 8)$ given that $(X_2 < 4)$ has occurred.
$\Pr{\brak{\brak{X_1 + X_2 = 8}|\brak{X_2 < 4}}}$
\begin{align}
=\frac{\Pr{((X_1 + X_2 = 8)((X_2 = 2)+(X_2 = 3)))}}{\Pr{\brak{(X_2 = 2)+(X_2 = 3)}}}
\end{align}
We have that,
\begin{align}
\Pr{(X_1 + X_2 = n)} &= 
\begin{cases}
0 & n < 1
\\
\frac{n-1}{36} &  2 \le n \le  7
\\
\frac{13-n}{36} & 7 < n \le 12
\\
0 & n > 12
\end{cases}
\end{align}
Therefore using equation (10) we can write that, 
\begin{align}
\Pr{(X_1 + X_2 = 8)} = \frac{5}{36}
\end{align}
From binomial distribution we can write ,
\begin{align}
\Pr{(X_2 < 4)}& = \Pr{(X_2=2)}+\Pr{(X_2=3)} \\ 
&= \binom{2}{1} \brak {\frac{1}{6}}\brak {\frac{5}{6}}+\binom{2}{2} \brak {\frac{1}{6}}^2\\
&= \frac{11}{36}
\end{align}
The event ${((X_1 + X_2 = 8)((X_2=2)+(X_2=3)))}$ is such that the sum is 8 with a number less than 4. \\
There are only two possible cases \{5,3\} and \{6,2\} out of 36 possible cases.\\
Hence,
\begin{align}
\Pr{((X_1 + X_2 = 8)((X_2=2)+(X_2=3)))}=\frac{2}{36}
\end{align}
Substituting equations (15) in (1), we get
\begin{align}
\begin{split}
\Pr{\brak{\brak{X_1 + X_2 = 8}|\brak{X_2 <  4}}} &= 
\frac{\frac{2}{36}}{\frac{1}{2}}\\
&=\frac{1}{9}
\end{split}
\end{align}  
Hence the probability of obtaining the sum 8 when a number is less than 4 is $\frac{1}{9}$. 

\end{enumerate}
\end{document}
